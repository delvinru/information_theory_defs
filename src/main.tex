\documentclass{article}

\usepackage[T2A]{fontenc}
\usepackage[utf8]{inputenc}
\usepackage[main=russian, english]{babel}

\usepackage{amsmath}

\title{Определения по Теории Информации}
\date{\today}
\author{Колесников Алексей}

\newcounter{defcntr}
\newcommand{\opr}{
    \stepcounter{defcntr}
    \emph{Определение \arabic{defcntr}.}
}

\begin{document}
\maketitle

\opr Дискретной случайной величиной $A$ называется определенное на вероятностном пространстве $(\Omega, F, P)$ и принимающим
значения из множества $a$, называется произвольное измеримое отображение $A: \Omega \rightarrow a$, т.е.
$$
\forall a_i \in a : A^{-1}(a_i) = \{\omega: A(\omega) = a_i\} \in F.
$$

\opr Вероятностная схема:
$$
A \sim
\begin{pmatrix}
    a_1 & a_2 & \dots \\
    p(a_1) & p(a_2) & \dots
\end{pmatrix}
$$
или
\begin{align*}
A \sim
\begin{pmatrix}
    a_1 & a_2 & \dots \\
    p(a_1) & p(a_2) & \dots
\end{pmatrix}
=
\begin{pmatrix}
    a_1 & a_2 & \dots & a_n \\
    p_1 & p_2 & \dots & p_n
\end{pmatrix}
=
( p_1, p_2, \dots, p_n ). \\
\forall i \in \overline{1, n} : p_i \geq 0, \sum_{i=1}^{n} p_i = 1.\\
B \sim
\begin{pmatrix}
    b_1 & b_2 & \dots \\
    p(b_1) & p(b_2) & \dots
\end{pmatrix}
=
\begin{pmatrix}
    b_1 & b_2 & \dots & b_m \\
    q_1 & q_2 & \dots & q_m
\end{pmatrix}
=
( q_1, q_2, \dots, q_m ). \\
\forall i \in \overline{1, m} : p_i \geq 0, \sum_{i=1}^{m} q_i = 1.
\end{align*}

\opr Объединенная вероятностная схема $AB$ с множеством исходов $a_ib_j$:
$$
AB \sim 
\begin{pmatrix}
    a_1b_1 & a_1b_2 & \dots & a_nb_m \\
    p(a_1b_1) & p(a_1b_2) & \dots & p(a_nb_m)
\end{pmatrix}.
$$

При этом $\sum\limits_{i=1}^{n} p(a_ib_j) = q_j$ и $\sum\limits_{j=1}^{m}p(a_ib_j) = p_j$.

Если $\sum\limits_{j=1}^{m}\sum\limits_{i=1}^{n}p(a_ib_j) = 1$, то события $A$ и $B$ независимы.

\opr Собственная информация, содержащаяся в исходе $a_i \in A$:
$$
I(a_i) = \log_2{\frac{1}{p(a_i)}}.
$$

\opr Энтропия вероятностной схемы $A$ (средняя собственная информация) (усреднение собственной информации $I(a_i)$ по вероятностной схеме $A$):
$$
    H(A) = E_A I(a_i) = \sum_{i=1}^{n} p(a_i) \cdot I(a_i) = \sum_{i=1}^{n} p(a_i) \cdot \log_2{\frac{1}{p(a_i)}}\\
$$
Итого,
$$
    H(A) = - \sum_{i=1}^{n} p(a_i) \cdot \log_2{p(a_i)}.
$$

\opr Собственная информация, содержащаяся в исходе $a_jb_j \in AB$:
$$
I(a_ib_j) = \log_2{\frac{1}{p(a_ib_j)}}.
$$

\opr Энтропия объединенной вероятностной схемы $AB$ (усреднение $I(a_ib_j)$ по вероятностной схеме $AB$):
$$
H(AB) = \sum_{i=1}^{n}\sum_{j=1}^{m} p(a_ib_j)\cdot \log_2{\frac{1}{p(a_ib_j)}}.
$$
Итого,
$$
H(AB) = - \sum_{(i, j)} p(a_ib_j) \cdot \log_2{p(a_ib_j)}.
$$

\opr Условная собственная информация, содержащаяся в исходе $a_i \in A$ при условии реализации исхода $b_j \in B$:
$$
I(a_i / b_j) = \log_2{\frac{1}{p(a_i / b_j)}},\ \text{где}\  p(a_i/b_j) = \frac{p(a_ib_j)}{p(b_j)}.
$$

\opr Условная энтропия вероятностной схемы $A$ относительно исхода $b_j \in B$ (усреднение $I(a_i/b_j)$ по вероятностной схеме $A/b_j$):
$$
H(A/b_j) = E_{A/b_j} I(a_i/b_j) = \sum_{i=1}^{n}p(a_i/b_j) \cdot I(a_i/b_j).
$$
Итого,
$$
H(A/b_j) = - \sum_{i=1}^{n} p(a_i / b_j) \log_2{p(a_i / b_j)}.
$$

\opr Условная энтропия вероятностной схемы $A$ относительно вероятностной схемы $B$ (усреднение $I(a_i/b_j)$ по вероятностной схеме $AB$):
$$
H(A/B) = E_{AB} I(a_i/b_j) = \sum_{i=1}^{n} \sum_{j=1}^{m} p(a_ib_j) I(a_i/b_j).
$$
Итого,
$$
H(A/B) = - \sum_{i=1}^{n} \sum_{j=1}^{m} p(a_ib_j) \log_2(p(a_i/b_j)).
$$

\end{document}